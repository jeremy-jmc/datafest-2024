\documentclass{article}
\usepackage[margin=1in]{geometry}

\begin{document}

\title{Metodologia Datafest 2024}
\author{Equipo UTEC}

\maketitle

\tableofcontents

\section{EDA}

\section{Forecast}

\section{Optimizacion Desagregada}


Dado un número de cajeros $N$ (en este caso 700). Aplicar la optimización para cada cajero individualmente, dado que no existen restricciones entre cajeros. Asumiendo cualquier cajero optimizable.

\subsection{Variables de decisión}

\begin{itemize}
    \item Sea j = 1, 2, ..., T la numeracion de los dias de la semana. Donde \textbf{T = 7}.
    \item Sea P(j) una variable binaria que indica si esta permitido llenar el cajero seleccionado durante el dia j.
    \begin{itemize}
        \item 1 si está permitido abastecer el cajero durante el dia $j$.
        \item 0 si está \textbf{NO} permitido abastecer el cajero durante el dia $j$.
    \end{itemize}
    \item Sea C la capacidad de dinero que puede alojar el ATM seleccionado
    \begin{itemize}
        \item 1000000 si el cajero es de \textbf{tipo A}.
        \item 1300000 si el cajero es de \textbf{tipo B}.
    \end{itemize}
    \item Sea R el porcentaje de costo por transportar X cantidad de dinero.
    \begin{itemize}
        \item 0.1\% si el cajero es de \textbf{tipo A}.
        \item 0.15\% si el cajero es de \textbf{tipo B}.
    \end{itemize}
    \item Sea $W(j)$ la cantidad esperada de cash retirada para el ATM seleccionado, durante el dia j.
    \item Sea $S(j)$ la cantidad de cash restante para el ATM seleccionado, al final del dia j.
    \item Sea $X(j)$ la cantidad de dinero llenado para el ATM seleccionado, al inicio del dia j.
\end{itemize}

\subsection{Función objetivo y Restricciones}

Minimizar:

$$
\sum_{j=1}^{T} R * X(j) * P(j)
$$

sujeto a:

\begin{enumerate}
    \item El cajero no caiga por debajo del stock de seguridad (20\% de la capacidad del cajero).
        $$0.2 * C <= S(j) , \forall j$$
    \item El dinero abastecido al cajero no exceda a su capacidad. Incluyendo los abastecimientos y demandas del día.
        $$S(j-1) + X(j) - W(j) <= C$$
    \item El dinero restante del día sea lo restante del dia anterior sumado a lo llenado menos la demanda.
        $$S(j) = S(j-1) + X(j) - W(j) , \forall j$$
        Donde $S(i, 0)$ es el dinero inicial de cada ATM (dato de la prediccion).
    \item El dinero sea llenado en los días que le corresponda al cajero según su tipo (A o B).
        $$X(j) <= O * P(j) , \forall j$$
        Donde $O$ es un numero grande ($10^9$, escogido porque es mayor a la capacidad máxima de cualquier cajero, ya sea de tipo A o B)
\end{enumerate}



\section{Optimizacion Agrupada}

\subsection{Variables de decisión}

\begin{itemize}
    \item Sea i = 1, 2, ..., N el id del cajero. Donde N = 700 cajeros
    \item Sea j = 1, 2, ..., T la numeracion de los dias de la semana. Donde T = 7.
    \item Sea P(i, j) una variable binaria que indica si esta permitido llenar el cajero i durante el dia j.
    \item Sea C(i) la capacidad de dinero que puede alojar un ATM.
    \item Sea R(i) el porcentaje de costo por transportar X cantidad de dinero. % \\
    % R(i) = \{ \\
    %     0.1\%, si el cajero es de tipo A \\
    %     0.15\%, si el cajero es de tipo A \\
    % \}
    \item Sea W(i, j) la cantidad esperada de cash retirada para el i-esimo ATM, durante el dia j.
    \item Sea S(i, j) la cantidad de cash restante para el i-esimo ATM, al final del dia j.
    \item Sea X(i, j) la cantidad de dinero llenado para el i-esimo ATM, al inicio del dia j.
\end{itemize}

\subsection{Restricciones}

Minimizar:

$$
\sum_{i=1}^{N} \sum_{j=1}^{T} R(i) * X(i, j) * P(i, j)
$$

sujeto a:

\begin{enumerate}
    \item Los cajeros no caigan por debajo del stock de seguridad (20\% de la capacidad del cajero).
        $$0.2 * C(i) <= S(i, j) , \forall i, j$$
    \item El dinero abastecido al cajero no exceda a su capacidad. Incluyendo la demanda (que puede ser negativa).
        $$S(i, j-1) + X(i, j) - W(i, j) <= C(i)$$
    \item El dinero restante del j-esimo dia sea lo restante del dia anterior sumado a lo llenado menos la demanda.
        $$S(i, j) = S(i, j-1) + X(i, j) - W(i, j)$$
        Donde $S(i, 0)$ es el dinero inicial de cada ATM (dato de la prediccion).
    \item El dinero sea llenado en los dias $j$ que le corresponda al cajero $i$.
        $$X(i, j) <= O * P(i, j)$$
        Donde $O$ es un numero grande ($1e9$, escogido porque es mayor a la capacidad máxima de cualquier cajero, ya sea de tipo A o B)
\end{enumerate}


\end{document}
